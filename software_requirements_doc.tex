% !TEX TS-program = pdflatex
% !TEX encoding = UTF-8 Unicode

% This is a simple template for a LaTeX document using the "article" class.
% See "book", "report", "letter" for other types of document.

\documentclass[11pt]{article} % use larger type; default would be 10pt

\usepackage[utf8]{inputenc} % set input encoding (not needed with XeLaTeX)

%%% Examples of Article customizations
% These packages are optional, depending whether you want the features they provide.
% See the LaTeX Companion or other references for full information.

%%% PAGE DIMENSIONS
\usepackage{geometry} % to change the page dimensions
\geometry{a4paper} % or letterpaper (US) or a5paper or....
% \geometry{margin=2in} % for example, change the margins to 2 inches all round
% \geometry{landscape} % set up the page for landscape
%   read geometry.pdf for detailed page layout information

\usepackage{graphicx} % support the \includegraphics command and options

% \usepackage[parfill]{parskip} % Activate to begin paragraphs with an empty line rather than an indent

%%% PACKAGES
\usepackage{booktabs} % for much better looking tables
\usepackage{array} % for better arrays (eg matrices) in maths
\usepackage{paralist} % very flexible & customisable lists (eg. enumerate/itemize, etc.)
\usepackage{verbatim} % adds environment for commenting out blocks of text & for better verbatim
\usepackage{subfig} % make it possible to include more than one captioned figure/table in a single float
% These packages are all incorporated in the memoir class to one degree or another...

%%% HEADERS & FOOTERS
\usepackage{fancyhdr} % This should be set AFTER setting up the page geometry
\pagestyle{fancy} % options: empty , plain , fancy
\renewcommand{\headrulewidth}{0pt} % customise the layout...
\lhead{}\chead{}\rhead{}
\lfoot{}\cfoot{\thepage}\rfoot{}

%%% SECTION TITLE APPEARANCE
\usepackage{sectsty}
\allsectionsfont{\sffamily\mdseries\upshape} % (See the fntguide.pdf for font help)
% (This matches ConTeXt defaults)

%%% ToC (table of contents) APPEARANCE
\usepackage[nottoc,notlof,notlot]{tocbibind} % Put the bibliography in the ToC
\usepackage[titles,subfigure]{tocloft} % Alter the style of the Table of Contents
\renewcommand{\cftsecfont}{\rmfamily\mdseries\upshape}
\renewcommand{\cftsecpagefont}{\rmfamily\mdseries\upshape} % No bold!

%%% END Article customizations

%%% The "real" document content comes below...

\usepackage{hyperref}

\title{Software Requirements Specification}
\author{Nikhil Patel}
%\date{} % Activate to display a given date or no date (if empty),
         % otherwise the current date is printed 

\begin{document}
\maketitle

\section{Overview}

\subsection{Purpose}
A dictionary is a very useful tool when learning a new language; similarly, a compilation of 
specialized vocabulary is useful for persons entering or employed in any profession. Having a 
consolidated, standardized reference for technical vocabulary helps prevent misinterpretation, 
which especially in professions such as healthcare can be disastrous. The main objective is to 
create a multipurpose jargon dictionary that would contain all such terminology, its 
contextualized definitions, and its accepted usage(s).
\subsection{Scope}
The dictionary is being made available as an open-source desktop application using an open dataset made available by the Government of Canada as Termium PLUS\textsuperscript{\textregistered}, the Government of Canada's terminology and linguistic databank\cite{termium}.

\section{Functional Requirements}

The functional requirements of a system are derived the fundamental reason for the product's existence. They detail what the system must do, but not how it is to be achieved.

\begin{enumerate}[{VP}1.]
	\item User
	\begin{enumerate}[{BE1}.1]
		\item User wants to search for a term
		\begin{enumerate}
			\item The interface must provide an interactive input widget
			\item The system should indicate the result of a search to the user
			\item If more than one entry is found the system should allow the user to refine his/her search
			\item The system must provide a means for navigating between entries
			\item If the database contains the word:
			\begin{enumerate}
			 	\item The system must display the definition and other relevant information
				\item The system must retrieve all occurrences of the word within the scope of the search
			\end{enumerate}
			\item If the word cannot be found in the database:
			\begin{enumerate}
				\item The system should display a number of closest lexical matches
				\item If no similar entries are found the system should suggest searching a different term or broadening the 					scope of the search
			\end{enumerate}
		\end{enumerate}
		\item User wants to search for a term within a category (optionally subject)
		\begin{enumerate}
			\item The system must provide the means to narrow the scope of a search to a subject
			\item If no entries are found the system should prompt the user to broaden his/her search
		\end{enumerate}
	\end{enumerate}
%	\item Developer
%	\begin{enumerate}[{BE2}.1]
%		\item Business Event
%		\begin{enumerate}
%			\item Requirement
%			\item Requirement
%			\item \dots
%		\end{enumerate}
%		\item Business Event
%		\begin{enumerate}
%			\item Requirement
%			\item Requirement
%			\item \dots
%		\end{enumerate}
%		\item \dots
%	\end{enumerate}
\end{enumerate}


\section{Non-functional Requirements}

%\subsection{Look and Feel Requirements}

%\subsection{Usability and Humanity Requirements}

%\subsection{Performance Requirements}

%\subsection{Operational and Environmental Requirements}

%\subsection{Installability Requirements}

%\subsection{Maintainability and Support Requirements}

%\subsection{Security Requirements}

%\subsection{Cultural Requirements}

%\subsection{Political Requirements}

\subsection{Legal Requirements}

The source for this project is licensed according to the GPLv3 license.

%\subsection{Health and Safety Requirements}


%\section{Anticipated Changes}


%\clearpage

\bibliographystyle{plain}
\bibliography{software_requirements_doc}

\end{document}